\documentclass{article}
\usepackage[utf8]{inputenc}
\begin{document}
\section{ Se eu fosse um dia o teu olhar }
Frio, o mar\\
Por entre o corpo\\
Fraco de lutar.\\
Quente, O chão\\
Onde te estendo\\
E te levo a razão.\\
Longa a noite\\
E só o sol\\
Quebra o silêncio,\\
Madrugada de cristal.\\
Leve, lento, nu, fiel\\
E este vento\\
Que te navega na pele.\\
Pede-me a paz\\
Dou-te o mundo\\
Louco, livre assim sou eu\\
(Um pouco mais...)\\
Solta-te a voz lá do fundo,\\
Grita, mostra-me a cor do céu.\\
\\
Se eu fosse um dia o teu olhar,\\
E tu as minhas mãos também,\\
Se eu fosse um dia o respirar\\
E tu perfume de ninguém.\\
Se eu fosse um dia o teu olhar,\\
E tu as minhas mãos também,\\
Se eu fosse um dia o respirar\\
E tu perfume de ninguém.\\
\\
Sangue, Ardente,\\
Fermenta e torna aos\\
Dedos de papel.\\
Luz, Dormente,\\
Suavemente pinta o teu rosto a\\
pincel.\\
Largo a espera,\\
E sigo o sul,\\
Perco a quimera\\
Meu anjo azul.\\
Fica, forte, sê amada,\\
Quero que saibas\\
Que ainda não te disse nada.\\
Pede-me a paz\\
Dou-te o mundo\\
Louco, livre assim sou eu\\
(Um pouco mais...)\\
Solta-te a voz lá do fundo,\\
Grita, mostra-me a cor do céu.\\
\\
Se eu fosse um dia o teu olhar,\\
E tu as minhas mãos também,\\
Se eu fosse um dia o respirar\\
E tu perfume de ninguém.\\
Se eu fosse um dia o teu olhar,\\
E tu as minhas mãos também,\\
Se eu fosse um dia o respirar\\
E tu perfume de ninguém.\\
\section{ Será}
Será que ainda me resta tempo\\
contigo,\\
ou já te levam balas de um qualquer\\
inimigo.\\
Será que soube dar-te tudo o que\\
querias,\\
ou deixei-me morrer lento, no lento\\
morrer dos dias.\\
Será que fiz tudo que podia fazer,\\
ou fui mais um cobarde, não quis ver\\
sofrer.\\
Será que lá longe ainda o céu é azul,\\
ou já o negro cinzento confunde Norte\\
com Sul.\\
Será que a tua pele ainda é macia,\\
ou é a mão que me treme, sem ardor\\
nem magia.\\
Será que ainda te posso valer,\\
ou já a noite descobre a dor que\\
encobre o prazer.\\
Será que é de febre este fogo,\\
este grito cruel que da lebre faz lobo.\\
Será que amanhã ainda existe para ti,\\
ou ao ver-te nos olhos te beijei e\\
morri.\\
Será que lá fora os carros passam\\
ainda,\\
ou estrelas caíram e qualquer sorte é\\
benvinda.\\
Será que a cidade ainda está como\\
dantes\\
ou cantam fantasmas e bailam\\
gigantes.\\
Será que o sol se põe do lado do mar,\\
ou a luz que me agarra é sombra de\\
luar.\\
Será que as casas cantam e as pedras\\
do chão,\\
ou calou-se a montanha, rendeu-se o\\
vulcão.\\
\\
Será que sabes que hoje é domingo,\\
ou os dias não passam, são anjos\\
caindo.\\
Será que me consegues ouvir\\
ou é tempo que pedes quando tentas\\
sorrir.\\
Será que sabes que te trago na voz,\\
que o teu mundo é o meu mundo e foi\\
feito por nós.\\
Será que te lembras da côr do olhar\\
quando juntos a noite não quer acabar.\\
Será que sentes esta mão que te agarra\\
que te prende com a força do mar\\
contra a barra.\\
Será que consegues ouvir-me dizer\\
que te amo tanto quanto noutro dia\\
qualquer.\\
Eu sei que tu estarás sempre por mim\\
não há noite sem dia, nem dia sem fim.\\
Eu sei que me queres, e me amas\\
também\\
me desejas agora como nunca\\
ninguém.\\
Não partas então, não me deixes\\
sozinho\\
Vou beijar o teu chão e chorar o\\
caminho.\\
Será,\\
Será,\\
Será!\\
\section{ Socorro}
Já não como à cinco dias\\
Não durmo à mais de um mês\\
Desde que te conheci\\
A minha vida é como vês.\\
Passo os dias a pensar\\
Não sei o que fazer,\\
Eu nem quero acreditar\\
No que me foi acontecer.\\
Só queria estar sozinho\\
E não pensar mais em amor,\\
Sempre que conhece alguém\\
Fico de mal a pior.\\
Li no "Metro" o teu anúncio de carácter pessoal\\
Limitavas-te a dizer... \\
\\
Curioso como sou\\
Apressei-me a responder,\\
Só para te perguntar\\
O que é que isso quer dizer.\\
Guardei o jornal no bolso\\
Para te falar depois,\\
Mas decorei o teu número\\
937812\\
Liguei-te às seis da tarde,\\
Dizias estar a acordar,\\
Essa voz rouca e quente\\
Num suave murmurar.\\
Fiquei quase sem fala,\\
Estive mesmo a desligar.\\
Do outro lado dizias... \\
\\
Socorro!! Estou a apaixonar-me\\
É impossível resistir a tanto charme. \\
\\
Foste-me buscar de carro\\
Levaste-me à beira mar,\\
Nas tuas mãos a 4L\\
Mais parece um jaguar!\\
Sentados na esplanada\\
A tomar um cimbalino,\\
Foi então que percebi\\
Essa coisa do destino.\\
Nesse dia aconteceu\\
Nunca mais vou esquecer\\
- o mar, o sol, o céu, a praia -\\
Todo um mundo de prazer,\\
Acendes um cigarro\\
Afagas-me o cabelo,\\
Disseste então assim... \\
\\
Não percebo o que é que queres\\
Nem o que estás a dizer,\\
Só sei que tu consegues\\
Mostrar o que é ser mulher,\\
Quando nós nos separamos\\
Não nos vimos por um mês,\\
Trinta dias a pensar\\
Em te ter mais uma vez\\
Depois vi-te na indústria\\
A dançar ao som de prince\\
Senti-me devorado\\
Pelo teu olhar de lince\\
Com ar discreto e decidido\\
Chegaste ao pé de mim,\\
Sussurraste-me ao ouvido... \\
\\
Socorro!! Estou a apaixonar-me\\
É impossível resistir a tanto charme. \\
\\
Encontrei-te então na baixa\\
(sem nada que o justifique)\\
Ali ficámos toda a tarde\\
Nos sofás do Magestic\\
Falaste-me do mundo\\
D' outras terras e lugares,\\
Mostras-me perfumes\\
De oceanos e mares.\\
Ali sentado viajei,\\
Ali para sempre quis ficar,\\
Contigo perto dos olhos\\
Os lábios quase a beijar.\\
Falaste da cidade,\\
Casas, ruas e pessoas\\
E disseste sem vaidade... \\
\\
Tenho ouvido muita coisa\\
Mas nunca tão bela assim,\\
Seduzir e encantar\\
São coisas novas p'ra mim.\\
O que eu gosto mais contigo\\
«se queres saber o que eu acho»\\
É que eu consigo ser homem,\\
Sem dar uma de macho.\\
Já não como à cinco dias\\
Não durmo à mais de um mês\\
Desde que te conheci\\
A minha vida é como vês.\\
Passo os dias a pensar\\
Não sei o que fazer,\\
Eu nem quero acreditar\\
No que me foi acontecer. \\
\\
Socorro!! Estou a apaixonar-me\\
É impossível resistir a tanto charme. \\
\section{ Tudo o que eu te dou}
Eu não sei, que mais posso ser\\
um dia rei, outro dia sem comer\\
por vezes forte, coragem de leão\\
as vezes fraco assim i o coração\\
eu não sei, que mais te posso dar\\
um dia jóias noutro dia o luar\\
gritos de dor, gritos de prazer\\
que um homem também chora\\
quando assim tem de ser\\
\\
Foram tantas as noites\\
sem dormir\\
tantos quartos de hotel\\
amar i partir\\
promessas perdidas\\
escritas no ar\\
e logo ali eu sei...\\
\\
Tudo o que eu te dou\\
tu de das a mim\\
tudo o que eu sonhei\\
tu serás assim\\
tudo o que eu te dou\\
tu me das a mim\\
tudo o que eu te dou\\
\\
Sentado na poltrona, beijas-me a pele morena\\
fazes aqueles truques que, aprendes-te no cinema\\
+, pego-te eu, já me sinto a viajar\\
para, recomeça, faz-me acreditar\\
\\
Não dizes tu, e o teu olhar mentiu\\
enrolados pelo chão no abraço que se viu\\
i madrugada ou i alucinação\\
estrelas de mil cores extasy ou paixão\\
hum, esse odor, traz tanta saudade\\
mata-me de amor ou da-me liberdade\\
deixa-me voar, cantar, adormecer\\
\\
refrão\\
\section{ * Viagens}
_ _ _ _ (2x)     C F Bb C \\
\\
_ _ _ _ (2x com saxofone)     C F Bb C\\
\\
Já vai alta a noite, vejo o _negro do ceu      F\\
De_itado na areia o teu _corpo e o meu      Bb C\\
viajo com as maos por entre as _montanhas e os rios      F \\
e _sinto nos meus lábios os _teus doces e frios      Bb C\\
\\
_E voas sobre o _mar, com as asas que eu te _dou     b Ab Eb \\
E dizes-me a ca_ntar: 'e' assim que eu s_ou'      Ab Eb\\
E olhar para ti, e ver o que eu _vejo      Ab\\
olhar-te nos olhos com olhares de d_esejo      Eb\\
E olhar para ti e ver o que eu _vejo      Ab\\
olhar-te nos olhos com olhares de d_esejo      Fm\\
Eu nao tenho nada _mais para te dar      Bb \\
esta _vida sao dois dias      C# \\
e _um e' para acordar      Ab\\
das hi_storias de encantar      Fm \\
das hi_storias de encantar      C#\\
\\
_(2_x _com _saxofone)     C F Bb C \\
\\
_Viagens que se perdem no tempo     C \\
vi_agens sem principio nem fim      F \\
_beijos entregues ao vento     Bb \\
e am_or em mares de cetim      C\\
\\
gestos que riscam o ar\\
e olh_ares que trazem solidao      F\\
_pedras e praias e o ceu a bailar     Bb \\
os c_orpos que fogem do chao      C\\
\\
_E voas sobre o m_ar com as asas que eu te d_ou     Eb Ab Eb\\
e dizes-me a cant_ar: 'e' assim que eu so_u'      Ab Eb \\
e olhar para ti e ver o que eu v_ejo      Ab\\
e olhar-te nos olhos com olhares de des_ejo      Eb\\
e olhar para ti e ver o que eu v_ejo      Ab\\
e olhar-te nos olhos com olhares de des_ejo      Fm \\
eu nao tenho nada m_ais para te dar      Bb \\
esta _vida sao dois dias e _um e' para acordar      C# Ab \\
das his_torias de encantar      Fm\\
das his_torias de encantar      C#\\
\\
[solo saxofone]\\
_      C\\
\section{Lira}
_Morte que mataste L_ira,           A       E7\\
Morte que mataste L_ira,            A\\
_Morte que mataste L_ira, _         A7      D       Bm\\
_Mata-me a m_im, que sou t_eu!        A       E7      A\\
\\
{soc}\\
Morte que mataste Lira\\
Mata-me a mim que sou teu\\
Mata-me com os mesmos ferros\\
Com que a Lira morreu\\
{eoc}\\
\\
A Lira por ser ingrata\\
Tiranamente morreu\\
A morte a mim não me mata\\
Firme e constante sou eu\\
\\
Veio um pastor lá da serra\\
À minha porta bateu\\
Veio me dar por notícia\\
Que a minha Lira morreu\\
\section{ * Os bravos}
_Eu fui à terra do _bravo   Am E\\
Bravo meu _bem           Am\\
Para ver se embra_vecia   E\\
_Cada vez fiquei mais _manso   Am E\\
Bravo meu _bem  Am\\
Para a tua compa_nhia   E\\
\\
Eu fui à terra do bravo\\
Bravo meu bem \\
Com o meu vestido vermelho\\
O que eu vi de lá mais bravo\\
Bravo meu bem\\
Foi um mansinho coelho\\
\\
As ondas do mar são brancas\\
Bravo meu bem\\
E no meio amarelas\\
Coitadinho de quem nasce\\
Bravo meu bem\\
P'ra morrer no meio delas\\
\\
Eu fui à terra do bravo\\
Bravo meu bem\\
Para ver se embravecia\\
Quiz bem a quem me quer mal\\
Bravo meu bem\\
Quiz bem a quem me não queria\\
\\
htmlnota: Outras quadras<p>\\
Dizes o teu amor bravo,<br>\\
Bravo meu bem, não é mais do que o meu bem.<br>\\
É bravo, porque não quer,<br>\\
Bravo meu bem, que eu olhe pr'a mais ninguém.<p>\\
\\
Eu fui à terra do Bravo,<br>\\
Bravo meu bem, vestidinha de amarelo.<br>\\
De amores que não são firmes,<br>\\
Bravo meu bem, tenho medo que me pelo.<p>\\
\\
Esta moda diz que é bravo,<br>\\
Bravo meu bem, mas eu vou cantar o manso.<br>\\
Para ver se, mansamente,<br>\\
Bravo meu bem, o teu bem-querer alcanço.<p>\\
\\
Ó Bravo, três vezes bravo,<br>\\
Bravo meu bem, ó Bravo, hás-de amansar.<br>\\
Tudo o que é bravo se amansa,<br>\\
Bravo meu bem, também te hei-de apanhar.\\
\section{ E alegre se fez triste}
Aquela clara madrugada que\\
Viu lágrimas correrem no teu rosto\\
E alegre se fez triste como se\\
chovesse de repente em pleno Agosto\\
\\
Ela só viu meus dedos nos teus dedos\\
Meu nome no teu nome e demorados\\
Viu nossos olhos juntos nos segredos\\
Que em silêncio dissemos separados\\
\\
A clara madrugada em que parti \\
Só ela viu teu rosto olhando a estrada\\
Por onde o automóvel se afastava\\
\\
E viu que a pátria estava toda em ti\\
E ouviu dizer adeus essa palavra\\
Que fez tão triste a clara madrugada\\
Que fez tão triste a clara madrugada\\
\section{ * Canção com lágrimas}
[Em C B7 Em]\\
\\
[Em]Eu canto para ti o mês das giestas\\
O [D]mês de morte e crescimento ó meu a[G]migo\\
Como [Am]um cristal part[D#7dim]indo-se plan[Em]gente\\
No [C]fundo da me[B7]mória pertur[Em]bada\\
\\
Eu canto para ti o mês onde começa a mágoa\\
E um coração poisado sobre a tua ausência\\
Eu canto um mês com lágrimas e sol o grave mês\\
Em que os mortos amados batem à porta do poema\\
\\
Porque tu me disseste quem em dera em Lisboa\\
Quem me dera me Maio depois morreste\\
Com Lisboa tão longe ó meu irmão tão breve\\
Que nunca mais acenderás no meu o teu cigarro\\
\\
Eu canto para ti Lisboa à tua espera\\
Teu nome escrito com ternura sobre as águas\\
E o teu retrato em cada rua onde não passas\\
Trazendo no sorriso a flor do mês de Maio\\
\\
Porque tu me disseste quem em dera em Maio\\
Porque te vi morrer eu canto para ti \\
Lisboa e o sol, Lisboa com lágrimas\\
Lisboa à tua espera ó meu irmão tão breve\\
Eu canto para ti Lisboa à tua espera...\\
\section{ Canção da fronteira}
Moça tão formosa\\
não vi na fronteira\\
como uma ceifeira\\
que cantava, Rosa\\
\\
Foi em Barca d´Alva\\
quando o sol nascia\\
uma ceifeira cantava\\
cantando vertia\\
trovas na fronteira\\
quando o sol nascia\\
\\
A saia de chita\\
rosinha, limão\\
que coisa bonita \\
sobre o coração\\
nos ramos da luz \\
um fruto limão\\
\\
De foice na mão\\
suspensa de um sonho\\
mordendo dois bagos\\
rubros de medronho\\
seus olhos dois bagos\\
suspensos de um sonho\\
\\
Devia ser pobre\\
mas cantava Rosa\\
romã que se abria\\
na manhã formosa\\
Que canto que sonho \\
que engano de rosa\\
\\
Foi em Barca d´Alva\\
quando o sol nascia\\
uma ceifeira cantava\\
cantando vertia\\
trovas na fronteira \\
quando o sol nascia\\
\\
Moça tão formosa \\
não vi na fronteira\\
como uma ceifeira\\
que cantava Rosa\\
\\
\section{ Canção terceira}
Quando desembarcarmos no Rossio canção\\
Vão dizer que a rua não é um rio\\
Vão apresar o teu navio \\
Carregado de vento carregado de pão\\
\\
Dirão que trazes tempestades\\
Dirão que vens de espada em riste\\
Dirão que foi sangue o vinho que pediste\\
Quando desembarcarmos no Rossio\\
\\
Vão vestir-te com grades\\
Que é um vestido para todas as idades \\
Na pátria dos poetas em Rossio triste\\
\\
Virão em busca do teu sonho e do teu pão\\
E vão exigir a nossa rendição \\
Mas eu canção\\
Eu gritarei de pé no teu navio\\
Não\\
\\
\\
Vão vestir-te com grades\\
Que é um vestido para todas as idades \\
Na pátria dos poetas em Rossio triste\\
Mas eu canção\\
Eu gritarei de pé no teu navio\\
Não\\
\section{ * Cantar de emigração (este parte, aquele parte)}
Este par_te, aquele par_te     Em G\\
e to_dos, to_dos se _vão       C Am D\\
Gali_za fi_cas sem ho_mens     D C Em\\
que po_ssam co_rtar teu pão_   C Am h7\\
_ _ _ _ _ _                        C C h7+ C C h7\\
\\
Tens em troca\\
órfãos e órfãs\\
tens campos de solidão\\
tens mães que não têm filhos\\
filhos que não têm pai\\
\\
Coração\\
que tens e sofre\\
longas ausências mortais\\
viúvas de vivos mortos\\
que ninguém consolará\\
\section{ Erguem-se muros}
Erguem-se muros em volta\\
do corpo quando nos damos\\
amor semeia a revolta\\
que nesse instante calamos\\
\\
Semeia a revolta e o dia \\
cobrir-se-á de navios (bis)\\
há que fazer-nos ao mar\\
antes que sequem os rios\\
\\
Secos os rios a noite\\
tem os caminhos fechados (bis)\\
Há que fazer-nos ao mar\\
ou ficaremos cercados\\
\\
Amor semeia a revolta\\
antes que sequem os rios...\\
\section{ As mãos}
Com mãos se faz a paz se faz a guerra.\\
Com mãos tudo se faz e se desfaz.\\
Com mãos se faz o poema - e são de terra.\\
Com mãos se faz a guerra - e são a paz.\\
\\
Com mãos se rasga o mar. Com mãos se lavra.\\
Não são de pedras estas casas, mas\\
de mãos. E estão no fruto e na palavra\\
as mãos que são o canto e são as armas.\\
\\
E cravam-se no tempo como farpas\\
as mãos que vês nas coisas transformadas.\\
Folhas que vão no vento: verdes harpas.\\
\\
De mãos é cada flor, cada cidade.\\
Ninguém pode vencer estas espadas:\\
nas tuas mãos começa a liberdade.\\
\\
\section{ * Menina dos olhos tristes}
_Menina dos olhos tristes             Em\\
_o que tanto a faz _chorar            D Em \\
_o soldadinho não _volta              G Am\\
_do outro lado do _mar _ _ _          h7 Em Em h7 Em\\
\\
Vamos senhor pensativo\\
olhe o cachimbo a apagar\\
o soldadinho não volta\\
do outro lado do mar\\
\\
Senhora de olhos cansados\\
porque a fatiga o tear\\
o soldadinho não volta\\
do outro lado do mar\\
\\
Anda bem triste um amigo\\
uma carta o fez chorar\\
o soldadinho não volta\\
do outro lado do mar\\
\\
A lua que é viajante\\
é que nos pode informar\\
o soldadinho já volta\\
está mesmo quase a chegar\\
\\
Vem numa caixa de pinho\\
do outro lado do mar\\
desta vez o soldadinho\\
nunca mais se faz ao mar\\
\\
htmlnote: \\
Zeca Afonso: "Menina dos Olhos Tristes"   Single Orfeu STAT-803   1969\\
Zeca Afonso: "De Capa e Batina"   CD Movieplay JA-8000    1996\\
\section{ Pensamento}
{soc}\\
Meu pensamento\\
partiu no vento\\
podem prendê-lo\\
matá-lo não{eoc}\\
\\
Meu pensamento\\
quebrou amarras\\
partiu no vento\\
deixou guitarras\\
meu pensamento\\
por onde passa\\
estátua de vento\\
em cada praça\\
\\
[Refrão]\\
\\
Foi à onquista\\
de um novo mundo\\
foi vagabundo\\
contrbandista\\
foi marinheiro\\
maltês ganhão\\
foi prisioneiro \\
mas servo não\\
\\
[Refrão]\\
\\
E os reis mandaram\\
fazer muralhas \\
tecer as malhas\\
de negras leis\\
homens morreram \\
estátuas ao vento\\
por ti morreram\\
meu pensamento\\
\section{ *Porque}
_Porque os outros se mascaram mas tu não   Am\\
Porque os _outros usam a vir_tude          G Am\\
Para com_prar o que não tem per_dão        C G\\
Porque os _outros têm medo mas tu _não _ _ _    Am Em    Am G Am\\
\\
Porque os outros são os túmulos caiados\\
Onde germina calada a podridão.\\
Porque os outros se calam mas tu não.\\
\\
Porque os outros se compram e se vendem\\
E os seus gestos dão sempre dividendo.\\
Porque os outros são hábeis mas tu não.\\
\\
Porque os outros vão à sombra dos abrigos\\
E tu vais de mãos dadas com os perigos.\\
Porque os outros calculam mas tu não.\\
\section{ Emigração (Quando no silêncio das noites de luar)}
Quando no silêncio das noites de luar\\
ia uma estrela pelos céus a correr \\
dizia minha mãe de mãos erguidas [bis]\\
Deus te salve por bem [bis]\\
\\
Desde então quando vejo que um homem\\
deixa a terra onde infeliz nasceu\\
e fortuna busca noutras praias digo [bis]\\
que te leve Deus também [bis]\\
\\
Não o acuso coitado não o acuso\\
nem lhe rogo pragas nem castigos\\
nem de que é dono de escolher, me esqueço [bis]\\
o que lhe convier [bis]\\
\\
Porque quem deixa o seu país natal\\
e fora dos seus caminhos põe os pés\\
e se troca o certo pelo incerto [bis]\\
motivos há-de ter [bis]\\
\\
\section{ * Tejo que levas as águas}
_Tejo que levas as águas            Em\\
_correndo de par em _par             D G\\
lava a cidade de _mágoas             C\\
leva as _mágoas para o _mar        Am h7\\
\\
Lava-a de crimes espantos\\
de roubos, fomes, terrores,\\
lava a cidade de quantos\\
do ódio fingem amores\\
\\
Leva nas águas as grades\\
de aço e silêncio forjadas\\
deixa soltar-se a verdade\\
das bocas amordaçadas\\
\\
Lava bancos e empresas\\
dos comedores de dinheiro\\
que dos salários de tristeza\\
arrecadam lucro inteiro\\
\\
Lava palácios vivendas\\
casebres bairros da lata\\
leva negócios e rendas \\
que a uns farta e a outros mata\\
\\
_Tejo que levas as _águas             C D\\
correndo de par em _par                 G\\
lava a cidade de _mágoas                C\\
leva as _mágoas para o _mar         Am h7\\
\\
Lava avenidas de vícios\\
vielas de amores venais\\
lava albergues e hospícios\\
cadeias e hospitais\\
\\
Afoga empenhos favores\\
vãs glórias, ocas palmas\\
leva o poder dos senhores\\
que compram corpos e almas\\
\\
Leva nas águas as grades\\
...\\
\\
Das camas de amor comprado\\
desata abraços de lodo\\
rostos corpos destroçados\\
lava-os com sal e iodo\\
\\
Tejo que levas nas águas\\
...\\
\section{ * Trova do vento que passa}
_Per_gunto ao vento que _passa       C G  C\\
no_tíci_as do meu pa_ís _              G G7 C C7\\
e o _vento _cala a des_graça_         F Fm C Am \\
o _vento _nada me _diz. _             Dm G C C7\\
o _vento _nada me _diz.              Dm G C\\
\\
_ _ _                               D A7 D\\
La-ra-lai-lai-lai-la, la-ra-lai-lai-lai-la, [Refrão]\\
La-ra-lai-lai-lai-la, la-ra-lai-lai-lai-la. [Bis]\\
\\
\\
Pergunto aos rios que levam\\
tanto sonho à flor das águas\\
e os rios não me sossegam\\
levam sonhos deixam mágoas.\\
\\
Levam sonhos deixam mágoas\\
ai rios do meu país\\
minha pátria à flor das águas\\
para onde vais? Ninguém diz.\\
\\
Mas há sempre uma candeia\\
dentro da própria desgraça\\
há sempre alguém que semeia\\
canções no vento que passa.\\
\\
Mesmo na noite mais triste\\
em tempo de servidão\\
há sempre alguém que resiste\\
há sempre alguém que diz não.\\
nota:\\
Alternativamente A7/D/G (refrão em D/A7) ou G7/C/F (refrão em C/G7)\\
\section{ O cantor de serviço}
Vêm de veludo encarnado\\
No andar um mistério qualquer\\
Nem discreto, nem ousado\\
O rosto bem aparado\\
Com sapatos de mulher\\
\\
Tudo é fogo, tudo é mar\\
Quando as luzes mudam de cor\\
Tudo cabe no mesmo lugar\\
Sortilégios do cantor\\
\\
Podem pedir canções\\
Eu faço por cantar\\
Escolham as ilusões\\
Nada mais posso dar\\
\\
Sobem de tom as cantigas\\
A orquestra por trás a puxar\\
São as grandes avenidas\\
Onde se vive outras vidas\\
A navegar, a navegar\\
\\
Tudo é sagrado e profano\\
Esta noite o encontro é total\\
Neste palco, mano a mano\\
Até ao compasso final\\
\\
Podem pedir canções\\
Eu faço por cantar\\
Escolham as ilusões\\
Nada mais posso dar\\
\\
\section{ * Fim do Mundo (e ao cabo do teu ser)}
[E]\\
[E]Vou alimentar a tua sede de querer vou\\
[F#-]assim cantar a tua [C#-]fome de prazer vou\\
[B]ao fim do [F#-]mundo [F#]\\
vou to[F#-]car lá no teu [B]fundo\\
\\
[E]Vou fechar o punho e pôr o sangue a ferver vou \\
[F#-]cerrar os dentes e mor[C#-]der o teu saber vou\\
[B]ao fim do [F#-]mundo [F#]\\
vou gri[F#-]tar lá no teu [B]fundo\\
\\
Sou [E]teu, sou teu\\
Sou [F#]teu, sou teu\\
[C#-]Sou assim [G#]só para cantar\\
[Fº7]e só a[F#-]ssim faz com que eu vá\\
[B]ao fim do [F#-]mundo [F#]ao fim ao [F#-]cabo do teu [B]ser\\
\\
[E]Sou e só apenas uma gota de suor sou\\
[F#-]um claro aceno quando-[C#-]o sou o tambor sou\\
[B]o fim do [F#-]mundo [F#]a con[F#-]tagem ao se[B]gundo\\
\\
[E]És todo o tempo que me resta a liberdade és\\
[F#-]a minha luta que só [C#-]fala com verdade és\\
[B]o fim do [F#-]mundo [F#]à en[F#-]trada na ci[B]dade\\
\\
[C#- G# Fº7 F#- A G# B F#- F# F#- B]\\
\\
Sou [E]teu, sou teu\\
Sou [F#]teu, sou teu\\
[C#-]Sou assim [G#]só para cantar\\
[Fº7]e só a[F#-]ssim faz com que eu vá\\
[B]ao fim do [F#-]mundo [F#]ao fim ao [F#-]cabo do teu [B]ser\\
\\
[E F#- C#- B F#- F# F#- B]\\
\\
[E]Vou fechar o punho e pôr o sangue a ferver vou \\
[F#-]cerrar os dentes e mor[C#-]der o teu saber vou\\
[B]ao fim do [F#-]mundo [F#]vou gri[F#-]tar lá no teu [B]fundo\\
\\
htmlnota: Acordes de viola\\
Aqui generalizo, o que consigo ouvir é 1xx101,que corresponde \\
apenas a Fº (1+3b+5b) e não Fº7 (1+3b+5b+7bb), que já tem o Ré.\\
Isto não é nada dogmático, tb posso estar a ouvir mal ;)\\
Uma alternativa seria 1x0101.\\
\section{ Há dias em que mais vale...}
Há dias\\
Em que não cabes na pele\\
Com que andas\\
Parece comprada em segunda mão\\
Um pouco curta nas mangas\\
\\
Há dias\\
Em que cada passo e mais um\\
Castigo de Deus\\
Parece\\
Que os sapatos que vês\\
Enfiados nos pés\\
Nem sequer são os teus\\
\\
A noite voltas a casa\\
Ao porto seguro\\
E p'ra sarar mais esta corrida\\
Vais lamber a ferida\\
Para o canto mais escuro\\
\\
Já vi\\
Há dias em que tu\\
não cabes em ti\\
\\
Avança\\
Na cara desse torpor\\
Que te perde e te seduz\\
A espada como a um Matador\\
Com o gesto maior\\
Do seu peito Andaluz\\
Avança\\
Com a raiva que sentes\\
Quando rangem os dentes\\
Ao peso da cruz\\
\\
Enfim,\\
Há dias em que eu\\
Também estou assim\\
\\
Parece que pagamos os\\
Pecados deste mundo\\
Amarrados aos remos de um \\
Barco que está no fundo.\\
\section{ Loucos de Lisboa}
Parava no café quando eu lá estava\\
Na voz tinha o talento dos pedintes\\
Entre um cigarro e outro lá cravava\\
a bica, ao melhor dos seus ouvintes\\
\\
As mãos e o olhar da mesma cor\\
Cinzenta como a roupa que trazia\\
Num gesto que podia ser de amor\\
Sorria, e ao sorrir agradecia\\
\\
{soc}\\
São os loucos de Lisboa\\
Que nos fazem recordar\\
A Terra gira ao contrário\\
E os rios correm para o mar{eoc}\\
\\
Um dia numa sala do quarteto\\
Passou um filme lá do hospital\\
Onde o esquecido filmado no gueto\\
Entrava como artista principal\\
\\
Compramos a entrada p'ra sessão\\
Pra ver tal personagem no écran\\
O rosto maltratado era a razão\\
De ele não aparecer pela manhã\\
\\
[refrão]\\
\\
Mudamos muita vez de calendário\\
Como o café mudou de freguesia\\
Deixamos de tributo a quem lá pára\\
Um louco a fazer-lhe companhia\\
\\
E sempre a mesma posse o mesmo olhar\\
De quem não mede os dias que vagueam\\
Sentado la continua a cravar\\
Beijinhos as meninas que passeiam.\\
\\
[refrão]\\
\section{ A Menina e os valetes}
Vinham do fundo da noite\\
Vinham de trás do luar\\
Em cavalinhos com asas\\
Vinham para te matar\\
\\
Em formação dois a dois\\
Uns de luto carregado\\
E logo atrás outros dois\\
Do mais brilhante encarnado\\
\\
E tu fugias, fugias\\
Da sombra deles no chão\\
De capa ao vento pareciam\\
Muito maiores do que são\\
\\
Um queria o mealheiro\\
Outro o teu coração\\
Com paus o mais desordeiro\\
De espada o maior vilão\\
\\
No fim do bosque uma luz\\
Que até parece dizer\\
Corre, corre menina\\
Nunca pares de correr\\
\\
Era a tua salvação\\
O poço da eternidade\\
Lá não caem os bandidos\\
Por não haver gravidade\\
\\
No meio de estrelas te lanças\\
Numa espiral de algodão\\
Os valentes são lembranças\\
D'uma outra dimensão\\
\\
De mansinho vais pousando\\
Ao som de uma voz que chama\\
É a tua mãe chamando\\
À beira da tua cama\\
\\
\section{ O troca pingas}
Ao petisco era branquinho\\
(tinha as suas opções)\\
Tinto as outras refeições\\
Agora anda mais liberto\\
Porque só bebe palheto\\
Tinha as suas opções\\
\\
Ia sempre a Almameda\\
Apoiar o Sindicato.\\
Agora anda de fato,\\
Que esta vida não comporta\\
Bandeiras atrás da porta\\
Apoiar o Sindicato\\
\\
Diz que agora esta melhor\\
Arranjou um bom emprego\\
Já não tem nada no prego\\
E pensa tirar até\\
Um curso da CEE.\\
Arranjou um bom emprego.\\
\\
Se há gente que anda com galo\\
E não ganha p'ro bitoque\\
Pode ser que não lhe toque\\
"- Deus e pai e não se esquece\\
Quem não luta não merece\\
E não ganha p'ro bitoque!"\\
\\
Antes tinha um capacete\\
Pejado de autocolantes\\
Mas isso... era dantes.\\
Agora tem outros usos,\\
Um carro pago aos soluços\\
Pejado de autocolantes\\
\\
Vai deixar-se de palheto\\
E passar a laranjada,\\
A tasca vai ser mudada\\
Para servir em pratinhos\\
Com saladas e bolinhos\\
E passar a laranjada.\\
\section{ A história do Zé Passarinho}
Pala saída que tem\\
Da vadiagem alguém\\
Chamou-lhe o Zé Passarinho\\
Fala em verso e as mulheres\\
Ao fim de duas colheres\\
Leva-as no bico p'ró ninho\\
\\
Sabe os fados do Alfredo\\
Rima que até mete medo\\
Nesta função é doutor\\
Tem os tiques de fadista\\
Mão no bolso, lenço e risca\\
"Baixem a luz por favor!"\\
\\
Uma triste noite ao frio\\
Cantava-se ao desafio\\
Para aquecer as paixões\\
Quando um estranho se levanta\\
Para mostrar como se canta\\
Faz-se à Rosa dos Limões\\
\\
O povo ficou sentido\\
Com aquele destemido\\
... morrer engasgado!\\
Palavra puxa palavra\\
Desata tudo à estalada\\
Com o posto ali ao lado\\
\\
Nem foi preciso a carrinha\\
Tudo na sua perninha\\
Numa linda procissão\\
Das perguntas com carinho\\
Ficou preso o Passarinho\\
Só para investigação\\
\\
Nasce o dia atrás da Sé\\
E ninguém arreda pé\\
Nem por dó, nem por esmola\\
O povo ficou sentado\\
Para ouvir cantar o fado\\
Passarinho na gaiola\\
\\
\section{ * Não sei se mereço}
A[Sol]cabou-se a sorte [Do]\\
Começou o meu azar [Re]\\
Não cumpri a minha parte [Do]\\
Agora tenho que ir trabalhar [Sol]\\
\\
Acabou-se a boa vida\\
É preciso lutar\\
Já não tenho mais oportunidades\\
Vou recomeçar\\
\\
Vou começar uma vida nova\\
Ainda há tempo para mudar\\
Qualquer dia estou com os pés para a cova\\
E não quero acreditar\\
\\
N[Sol]ão sei se me[Do]reço\\
Esta [Sol]vida de [Do]cão\\
Tudo o que [Sol]vejo tem um [Do]preço\\
E eu não [Re]tenho um tostão\\
\\
Será que mereço\\
Nunca fiz mal a ninguém\\
Estou perdido não tenho nada\\
Vou chamar a minha mãe\\
\\
Sei que a vida não dura sempre\\
tempo passa devagar\\
Tenho tempo para rir\\
Tenho tempo para chorar\\
\\
Voltas e voltas\\
E que grande confusão\\
Um homem anda aqui à toa\\
Nem sinto os pés no chão\\
\\
Não sei se mereço\\
Esta vida de cão\\
Tudo o que vejo tem um preço\\
E eu não tenho um tostão\\
\\
Será que mereço\\
Nunca fiz mal a ninguém\\
Estou perdido não tenho nada\\
Vou chamar a minha mãe\\
\section{ alecrim}
Alecrim alecrim aos molhos\\
por causa de ti\\
choram os meus olhos\\
ai meu amor\\
quem te disse a ti\\
que a flor do monte\\
era o alecrim\\
\\
Alecrim alecrim doirado\\
que nasce no monte \\
sem ser semeado\\
ai meu amor\\
quem te disse a ti\\
que a flor do monte\\
era o alecrim\\
\section{ Canção tão simples (Quem poderá domar ...)}
Quem poderá domar os cavalos do vento\\
quem poderá domar este tropel\\
do pensamento\\
à flor da pele?\\
\\
Quem poderá calar a voz do sino triste\\
que diz por dentro do que não se diz\\
a fúria em riste\\
do meu país?\\
\\
Quem poderá proibir estas letras de chuva\\
que gota a gota escrevem nas vidraças\\
pátria viúva\\
a dor que passa?\\
\\
Quem poderá prender os dedos farpas\\
que dentro da canção fazem das brisas\\
as armas harpas\\
que são precisas?\\
\section{ Estranha forma de vida}
Foi por vontade de Deus\\
que eu vivo nesta ansiedade.\\
Que todos os ais são meus,\\
Que é toda a minha saudade.\\
Foi por vontade de Deus.\\
\\
Que estranha forma de vida\\
tem este meu coração:\\
vive de forma perdida;\\
Quem lhe daria o condão?\\
Que estranha forma de vida.\\
\\
Coração independente,\\
coração que não comando:\\
vive perdido entre a gente,\\
teimosamente sangrando,\\
coração independente.\\
\\
Eu não te acompanho mais:\\
para, deixa de bater.\\
Se não sabes aonde vais,\\
porque teimas em correr,\\
eu não te acompanho mais.\\
\\
\\
\section{ Libertação}
Fui à praia, e vi nos limos\\
a nossa vida enredada:\\
ó meu amor, se fugimos,\\
ninguém saberá de nada.\\
\\
Na esquina de cada rua,\\
uma sombra nos espreita,\\
e nos olhares se insinua,\\
de repente uma suspeita.\\
\\
Fui ao campo, e vi os ramos\\
decepados e torcidos:\\
ó meu amor, se ficamos,\\
pobres dos nossos sentidos.\\
\\
Hão-de transformar o mar\\
deste amor numa lagoa:\\
e de lodo hão-de a cercar,\\
porque o mundo não perdoa.\\
\\
Em tudo vejo fronteiras,\\
fronteiras ao nosso amor.\\
Longe daquí,onde queiras,\\
a vida será maior.\\
\\
Nem as esp'ranças do céu\\
me conseguem demover\\
Este amor é teu e meu:\\
só na terra o queremos ter.\\
\\
\\
\section{ Há festa na Mouraria}
Há festa na Mouraria,\\
é dia da procissão\\
da senhora da saúde.\\
Até a Rosa Maria\\
da rua do Capelão\\
parece que tem virtude.\\
\\
Naquele bairro fadista\\
calaram-se as guitarradas:\\
não se canta nesse dia,\\
velha tradição bairrista,\\
vibram no ar badaladas,\\
há festa na Mouraria.\\
\\
Colchas ricas nas janelas,\\
pétalas soltas no chão.\\
Almas crentes, povo rude\\
anda a fé pelas vielas:\\
é dia da procissão\\
da senhora da saúde.\\
\\
Após um curto rumor\\
profundo siléncio pesa:\\
por sobre o largo da guia\\
passa a Virgem no andor.\\
Tudo se ajoelha e reza,\\
até a Rosa Maria.\\
\\
Como que petrificada,\\
em fervorosa oração,\\
é tal a sua atitude,\\
que a rosa já desfolhada\\
da rua do Capelão\\
parece que tem virtude.\\
\\
\\
\section{ Não é desgraça ser pobre}
Não é desgraça ser pobre,\\
não é desgraça ser louca:\\
desgraça é trazer o fado\\
no coração e na boca.\\
\\
Nesta vida desvairada,\\
ser feliz é coisa pouca.\\
Se as loucas não sentem nada,\\
não é desgraça ser louca.\\
\\
Ao nascer trouxe uma estrela;\\
nela o destino traçado.\\
Não foi desgraça trazé-la:\\
desgraça é trazer o fado.\\
\\
Desgraça é andar a gente\\
de tanto cantar, já rouca,\\
e o fado, teimosamente,\\
no coração e na boca.\\
\\
\\
\\
\section{ Gaivota}
Se uma gaivota viesse\\
trazer-me o céu de Lisboa\\
no desenho que fizesse,\\
nesse céu onde o olhar\\
é uma asa que não voa,\\
esmorece e cai no mar.\\
\\
Que perfeito coração\\
no meu peito bateria,\\
meu amor na tua mão,\\
nessa mão onde cabia\\
perfeito o meu coração.\\
\\
Se um português marinheiro,\\
dos sete mares andarilho,\\
fosse quem sabe o primeiro\\
a contar-me o que inventasse,\\
se um olhar de novo brilho\\
no meu olhar se enlaçasse.\\
\\
Que perfeito coração\\
no meu peito bateria,\\
meu amor na tua mão,\\
nessa mão onde cabia\\
perfeito o meu coração.\\
\\
Se ao dizer adeus à vida\\
as aves todas do céu,\\
me dessem na despedida\\
o teu olhar derradeiro,\\
esse olhar que era só teu,\\
amor que foste o primeiro.\\
\\
Que perfeito coração\\
no meu peito morreria,\\
meu amor na tua mão,\\
nessa mão onde perfeito\\
bateu o meu coração.\\
\\
\\
\\
\section{ Com que voz}
Com que voz chorarei meu triste fado,\\
que em tão dura paixão me sepultou.\\
Que mor não seja a dor que me deixou\\
o tempo, de meu bem desenganado.\\
\\
Mas chorar não estima neste estado\\
aonde suspirar nunca aproveitou.\\
Triste quero viver, poi se mudou\\
em tisteza a alegria do passado.\\
\\
Assim a vida passo descontente,\\
ao som nesta prisão do grilhão duro\\
que lastima ao pé que a sofre e sente.\\
\\
De tanto mal, a causa é amor puro,\\
devido a quem de mim tenho ausente,\\
por quem a vida e bens dele aventuro.\\
\\
\\
\section{ Madrugada de Alfama}
Mora num beco de Alfama\\
e chamam-lhe a madrugada,\\
mas ela, de tão estouvada\\
nem sabe como se chama.\\
\\
Mora numa água-furtada\\
que é a mais alta de Alfama\\
e que o sol primeiro inflama\\
quando acorda à madrugada.\\
Mora numa água-furtada\\
que é a mais alta de Alfama.\\
\\
Nem mesmo na Madragoa\\
ninguém compete com ela,\\
que do alto da janela\\
tão cedo beija Lisboa.\\
\\
E a sua colcha amarela\\
faz inveja à Madragoa:\\
Madragoa não perdoa\\
que madruguem mais do que ela.\\
E a sua colcha amarela\\
faz inveja à Madragoa.\\
\\
Mora num beco de Alfama\\
e chamam-lhe a madrugada;\\
são mastros de luz doirada\\
os ferros da sua cama.\\
\\
E a sua colcha amarela\\
a brilhar sobre Lisboa,\\
é como a estatua de proa\\
que anuncia a caravela,\\
a sua colcha amarela\\
a brilhar sobre Lisboa.\\
\\
\\
\section{ Ai, Mouraria}
Ai, Mouraria\\
da velha Rua da Palma,\\
onde eu um dia\\
deixei presa a minha alma,\\
por ter passado\\
mesmo ao meu lado\\
certo fadista\\
de cor morena,\\
boca pequena\\
e olhar troçista.\\
\\
Ai, Mouraria\\
do homem do meu encanto\\
que me mentia,\\
mas que eu adorava tanto.\\
Amor que o vento,\\
como um lamento,\\
levou consigo,\\
mais que ainda agora\\
a toda a hora\\
trago comigo.\\
\\
Ai, Mouraria\\
dos rouxinóis nos beirais,\\
dos vestidos cor-de rosa,\\
dos pregões tradicionais.\\
Ai, Mouraria\\
das procissões a passar,\\
da Severa em voz saudosa,\\
da guitarra a soluçar.\\
\\
\\
\section{ Fado português}
O Fado nasceu um dia,\\
quando o vento mal bulia\\
e o céu o mar prolongava,\\
na amurada dum veleiro,\\
no peito dum marinheiro\\
que, estando triste, cantava,\\
que, estando triste, cantava.\\
\\
Ai, que lindeza tamanha,\\
meu chão , meu monte, meu vale,\\
de folhas, flores, frutas de oiro,\\
vê se vês terras de Espanha,\\
areias de Portugal,\\
olhar ceguinho de choro.\\
\\
Na boca dum marinheiro\\
do frágil barco veleiro,\\
morrendo a canção magoada,\\
diz o pungir dos desejos\\
do lábio a queimar de beijos\\
que beija o ar, e mais nada,\\
que beija o ar, e mais nada.\\
\\
Mãe, adeus. Adeus, Maria.\\
Guarda bem no teu sentido\\
que aqui te faço uma jura:\\
que ou te levo à sacristia,\\
ou foi Deus que foi servido\\
dar-me no mar sepultura.\\
\\
Ora eis que embora outro dia,\\
quando o vento nem bulia\\
e o céu o mar prolongava,\\
à proa de outro velero\\
velava outro marinheiro\\
que, estando triste, cantava,\\
que, estando triste, cantava.\\
\\
\\
\\
\section{ Maria Lisboa}
É varina, usa chinela,\\
tem movimentos de gata;\\
na canastra, a caravela,\\
no coração, a fragata.\\
\\
Em vez de corvos no chaile,\\
gaivotas vêm pousar.\\
Quando o vento a leva ao baile,\\
baila no baile com o mar.\\
\\
É de conchas o vestido,\\
tem algas na cabeleira,\\
e nas velas o latido\\
do motor duma traineira.\\
\\
Vende sonho e maresia,\\
tempestades apregoa.\\
Seu nome próprio: Maria;\\
seu apelido: Lisboa.\\
\section{ * Eu quero amar, amar perdidamente}
_Eu quero amar _amar perdidamente_          D   D7  Gm\\
_Amar só por amar aqui e além_              C   F\\
_Mais esta _aquela a outra e toda a gente_  D7  Gm Dm\\
_Amar amar e não amar _ninguêm _ _          A#  C   A#  A+\\
\\
_Recordar esquecer _é indiferente_          Dm  D7  Gm\\
_Prender ou desprender é mal é bem_         C   F+\\
_Quem disser que _se pode amar alguém_      D7  Gm Dm\\
_Durante a vida _inteira é porque mente_    A#  A+  D+\\
\\
_Há uma prima_vera em cada vida_            D+  B7  Em\\
É preciso cantá-la assim florida_           A+\\
Pois se Deus nos deu voz foi p´ra cantar_   D+\\
\\
E se um _dia hei-de ser pó cinza e nada_    B7  Em\\
Que seja a minha noite uma alvorada_        A+\\
Que me saiba perder p´ra me encontrar_      D+\\
\section{ Amêndoa Amarga}
Port ti falo\\
e ninguém pensa\\
mas eu digo\\
minha amêndoa, meu amigo\\
meu irmão\\
meu tropel de ternura\\
minha casa\\
meu jardim de carência\\
minha asa.\\
\\
Por ti vivo\\
e ninguém pensa\\
mas eu sigo\\
um caminho de silvas\\
e de nardos\\
uma intensa ternura\\
que persigo\\
rodeada de cardos\\
por tantos lados.\\
\\
Por ti morro\\
e ninguém sabe\\
mas eu espero \\
o teu corpo que sabe \\
a madrugada\\
o teu corpo que sabe\\
a desespero\\
\\
ó minha amarga amêndoa \\
desejada.\\
\\
ó minha amarga amêndoa\\
desejada.\\
\\
\section{ *Barco negro}
_De manhã, temendo que me ach_asses f_eia!    A	E7	A\\
_Acordei, tremendo, deitada n'ar_eia      A7	D\\
M_as logo os teus olhos diss_eram que n_ão,     A	A7	D\\
_E o sol penetr_ou no meu c_oraç_ão.[Bis]     A7	A	E7	A\\
\\
Vi depois, numa rocha, uma cr_uz,        A7\\
E o teu barco negro dançav_a na l_uz     E7	D\\
Vi t_eu braço acenando, entre as _velas já s_oltas     A	A7  D\\
D_izem as velhas da praia, _que não v_oltas:   A   E7	A\\
\\
São l_oucas! São l_oucas!      C	A\\
\\
Eu s_ei, meu amor,      E7\\
Que nem chegaste a part_ir,      A\\
Pois t_udo, em meu redor,     E7\\
Me diz qu'estás sempre com_igo.[Bis]      A\\
\\
No vento que lança areia nos vidros;\\
Na água que canta, no fogo mortiço;\\
No calor do leito, nos bancos vazios;\\
Dentro do meu peito, estás sempre comigo.\\
\\
\section{ Uma casa portuguesa}
Numa casa portuguesa fica bem\\
pão e vinho sobre a mesa.\\
Quando à porta humildemente bate alguém,\\
senta-se à mesa co'a gente.\\
Fica bem essa fraqueza, fica bem,\\
que o povo nunca a desmente.\\
A alegria da pobreza\\
está nesta grande riqueza\\
de dar, e ficar contente.\\
\\
Quatro paredes caiadas,\\
um cheirinho á alecrim,\\
um cacho de uvas doiradas,\\
duas rosas num jardim,\\
um São José de azulejo\\
sob um sol de primavera,\\
uma promessa de beijos\\
dois braços à minha espera...\\
É uma casa portuguesa, com certeza!\\
É, com certeza, uma casa portuguesa!\\
\\
No conforto pobrezinho do meu lar,\\
há fartura de carinho.\\
A cortina da janela e o luar,\\
mais o sol que gosta dela...\\
Basta pouco, poucochinho p'ra alegrar\\
uma existéncia singela...\\
É só amor, pão e vinho\\
e um caldo verde, verdinho\\
a fumegar na tijela.\\
\\
Quatro paredes caiadas,\\
um cheirinho á alecrim,\\
um cacho de uvas doiradas,\\
duas rosas num jardim,\\
um São José de azulejo\\
sob um sol de primavera,\\
uma promessa de beijos\\
dois braços à minha espera...\\
É uma casa portuguesa, com certeza!\\
É, com certeza, uma casa portuguesa!\\
\\
\\
\\
\section{ Dar de beber à dor}
Foi no Domingo passado que passei\\
à casa onde vivia a Mariquinhas,\\
mas 'stá tudo tão mudado\\
que não vi em menhum lado\\
as tais janelas que tinham tabuinhas.\\
Do rés-do-chão ao telhado\\
não vi nada, nada, nada\\
que pudesse recordar-me a Mariquinhas,\\
e há um vidro pregado e azulado\\
onde havia as tabuinhas.\\
\\
Entrei e onde era a sala agora está\\
à secretária um sujeito que é lingrinhas,\\
mas não vi colchas com barra\\
nem viola, nem guitarra,\\
nem espreitadelas furtivas das vizinhas.\\
O tempo cravou a garra\\
na alma daquela casa\\
onde as vezes petiscavamos sardinhas\\
quando em noites de guitarra e de farra\\
estava alegre a Mariquinhas.\\
\\
As janelas tão garridas que ficavam\\
com cortinados de chita às pintinhas\\
perderam de todo a graça\\
porque é hoje uma vidraça\\
com cercadura de lata às voltinhas.\\
E lá p'ra dentro quem passa\\
hoje é p'ra ir aos penhores\\
entregar ao usurário umas coisinhas,\\
pois chega a esta desgraça toda a graça\\
da casa da Mariquinhas.\\
\\
P'ra terem feito da casa o que fizeram\\
melhor fora que a mandassem p'rás alminhas,\\
pois ser casa de penhores\\
o que foi viveiro d'amores\\
é ideia que não cabe cá nas minhas\\
recordações do calor \\
e das saudades. O gosto\\
que eu vou procurar esquecer\\
numas ginginhas,\\
pois dar de beber à dor é o melhor,\\
já dizia a Mariquinhas.\\
\section{ Dura Memória}
Memória do meu bem cortado em flores\\
por ordem de meus tristes e maus fados\\
deixai-me descansar com meus cuidados\\
nesta inquietação dos meus amores.\\
\\
Basta-me o mal presente e os temores\\
dos sucessos que espero infortunados\\
sem que venham de novo bens passados\\
à afrontar meu repouso com suas dores.\\
\\
Perdi e mora tudo quanto em termos\\
tão vagarosos e largos alcancei\\
deixai-me com as lembranças desta glória\\
deixai-me lembranças desta glória.\\
\\
Cumpre-se e acaba a vida nestes zelos\\
porque neles com meu baile a acabarei\\
porque neles com meu baile acabarei\\
mil vidas não, uma só dura memória.\\
\section{ Fado do Ciúme}
Se não esqueceste\\
o amor que me dedicaste\\
e o que escreveste\\
nas cartas que me mandaste\\
esquece o passado \\
e volta para meu lado\\
porque já estás perdoado\\
de tudo o que me chamaste.\\
\\
Volta meu querido\\
mas volta como disseste\\
arrependido\\
de tudo o que me fizeste,\\
haja o que houver\\
já basta p'ra teu castigo\\
essa mulher \\
que andava agora contigo.\\
\\
Se é contrafeito\\
não voltes toma cautela\\
porque eu aceito\\
que vivas antes com ela\\
pois podes crer\\
que antes prefiro morrer\\
do que contigo viver\\
sabendo que gostas dela.\\
\\
Só o que eu peço \\
é uma recordação\\
se é que mereço\\
um pouco de compaixão,\\
deixa ficar \\
o teu retrato comigo\\
p'ra eu julgar\\
que ainda vivo contigo.\\
\\
\section{ Fado Xuxu}
O fado, canção bizarra\\
pôs a samarra\\
todo trecheiro\\
e lá foi com a guitarra\\
até ao Rio de Janeiro.\\
\\
Fez-se um fadista atrevido\\
tão destemido\\
e de tal marca\\
que até já é conhedico\\
p'lo fadistão da Fuzarca.\\
\\
Com sambinhas\\
e modinhas\\
abacate\\
vitamate\\
Guaraná\\
maracujá\\
e caruru\\
\\
Com cocada\\
batucada\\
para ti \\
abacaxi\\
e goiabada\\
o fado é bom p'ra xuxu.\\
\\
Um portuguesinho de raça\\
bebe cachaça \\
come pipoca\\
e no catete até passa\\
por cidadão carioca.\\
\\
Às vezes vai à favela\\
calça chinela\\
todo se bamba...\\
e o fado canção singela \\
agora é todo do samba.\\
\section{ Grito}
Silêncio!\\
do silêncio fasso um grito\\
corpo todo me dói\\
deixai-me chorar um pouco.\\
\\
Só à sombra\\
como o sol vou rebolindo \\
de sombra assombrada\\
já lhe perdi o sentido.\\
\\
Ó céu!\\
aqui me falta a luz\\
aqui me falta uma estrela\\
chora-se mais \\
quando se vive atrás dela.\\
\\
E eu,\\
a quem o sol esqueceu\\
só dou ao mundo perdão \\
só choro agora \\
porque quem morre já não chora.\\
\\
Solidão!\\
que lembras-me a santeira\\
ao céu da companheira\\
minha profunda amargura.\\
\\
Ai, solidão\\
a quem foste confiante\\
Ai! solidão\\
e se morderam a cabeça.\\
\\
Meu Deus\\
que ao fim do além da vida\\
do que já fui tenho sede\\
sou sombra triste\\
encostada a uma parrede.\\
\\
Adeus, \\
vida que ranto duras\\
da morte que tanto gabas\\
ai, que me dês\\
a solidão quase loucura.\\
\\
\section{ Havemos de ir a Viana}
Entre sombras misteriosas\\
em rompendo ao longe estrelas\\
trocaremos nossas rosas\\
para depois esquecê-las.\\
\\
Se o meu sangue não me engana \\
como engana a fantasia \\
havemos de ir a Viana\\
ó meu amor de algum dia\\
ó meu amor de algum dia\\
havemos de ir a Viana\\
se o meu sangue não me engana \\
havemos de ir a Viana.\\
\\
Partamos de flor ao peito\\
que o amor é como o vento\\
quem pára perde-lhe o jeito\\
e morre a todo o momento.\\
\\
Se o meu sangue não me engana \\
como engana a fantasia\\
havemos de ir a Viana\\
ó meu amor de algum dia\\
ó meu amor de algum dia\\
havemos de ir a Viana\\
se o meu sangue não me engana\\
havemos de ir a Viana.\\
\\
Ciganos, verdes ciganos\\
deixai-me com esta crença\\
os pecados têm vinte anos \\
os remorços têm oitenta.\\
\section{ Nem às paredes confesso}
Não queiras gostar de mim\\
Sem que eu te peça,\\
Nem me dês nada que ao fim\\
Eu não mereça\\
Vê se me deitas depois\\
Culpas no rosto\\
Eu sou sincera\\
Porque não quero\\
Dar-te um desgosto\\
\\
[refrão:]\\
De quem eu gosto\\
nem às paredes confesso\\
E nem aposto\\
Que não gosto de ninguém\\
Podes rogar\\
Podes chorar\\
Podes sorrir também\\
De quem eu gosto\\
Nem às paredes confesso.\\
\\
Quem sabe se te esqueci\\
Ou se te quero\\
Quem sabe até se é por ti\\
que eu tanto espero.\\
Se gosto ou não afinal\\
Isso é comigo,\\
Mesmo que penses\\
Que me convences\\
Nada te digo.\\
\\
\section{ Povo que lavas no rio}
Povo que lavas no rio\\
Que talhas com o teu machado\\
As tábuas do meu caixão.\\
Pode haver quem te defenda\\
Quem compre o teu chão sagrado\\
Mas a tua vida não.\\
\\
Fui ter à mesa redonda\\
Beber em malga que me esconda\\
O beijo de mão em mão.\\
Era o vinho que me deste\\
Água pura, fruto agreste\\
Mas a tua vida não.\\
\\
Aromas de urze e de lama\\
Dormi com eles na cama\\
Tive a mesma condição.\\
Povo, povo, eu te pertenço\\
Deste-me alturas de incenso,\\
Mas a tua vida não.\\
\\
Povo que lavas no rio\\
E talhas com o teu machado\\
As tábuas do meu caixão.\\
Pode haver quem te defenda\\
Quem compre o teu chão sagrado\\
Mas a tua vida não.\\
\section{ Prece}
Talvez que eu morra na práia\\
cercada e perfido no banho\\
por toda a espuma da práia\\
como um pastor que desmaia\\
no meio do seu rebanho.\\
\\
Talvez que eu morra na rua\\
e dê por mim derrepente\\
em noite fria e sem luar\\
e mando as pedras da rua\\
pisadas por toda a gente.\\
\\
Talvez que eu morra entre grades\\
no meio de uma prisão\\
porque o mundo além das grades\\
venha esquecer as saudades\\
que roiem meu coração.\\
\\
Talvez que eu morra de noite\\
onde a morte é natural\\
as mãos em cruz sobre o peito\\
das mãos de Deus tudo aceito\\
mas que eu morra em Portugal\\
\\
\section{ Medo (Quem dorme à noite comigo?)}
Quem dorme à noite comigo?\\
É meu segredo, é meu segredo!\\
Mas se insistirem, desdigo.\\
O medo mora comigo,\\
Mas só o medo, mas só o medo!\\
\\
E cedo, porque me embala\\
Num vaivém de solidão,\\
É com silêncio que fala,\\
Com voz de móvel que estala\\
E nos perturba a razão.\\
\\
Que farei quando, deitado,\\
Fitando o espaço vazio,\\
Grita no espaço fitado\\
Que está dormindo a meu lado,\\
Lázaro e frio?\\
\\
Gritar? Quem pode salvar-me\\
Do que está dentro de mim?\\
Gostava até de matar-me.\\
Mas eu sei que ele há-de esperar-me\\
Ao pé da ponte do fim.\\
\section{ Raízes}
Velhas pedras que pisei\\
saiam da vossa mudez\\
venham dizer o que sei\\
venham falar português\\
sejam duras como a lei\\
e puras como a nudez.\\
\\
Minha lágrima salgada\\
caíu no lenço da vida\\
foi lembrança naufragada\\
e para sempre perdida\\
foi vaga despedaçada\\
contra o cais da despedida.\\
\\
Visitei tantos países\\
conheci tanto luar\\
nos olhos dos infelizes\\
e porque me hei-de gastar?\\
vou ao fundo das raízes \\
e hei-de gastar-me a cantar.\\
\section{ Puestos entan frente a frente }
Puestos entan frente a frente\\
Los dos valerosos campos,\\
Uno es del Rey Maluco,\\
Otro de Sebastiano\\
El Lusitano.\\
Moço, animoso y valiente,\\
Robusto, determinado,\\
Aunque de poca experiencia\\
Y no bien aconsejado,\\
El Lusitano.\\
\\
Brama que entrevistan los moros\\
Y el exercito contrario\\
Ya se vá llegando cerca\\
Aellos (dize) Santiago,\\
El Lusitano.\\
Dispara la ertelharia,\\
La nuestra mal disparando\\
Llueven balas, llueve muerte,\\
Saetas y mosquetazos.\\
El Lusitano.\\
\\
Que por los lados ya todos\\
Y con sangre de los muertos,\\
Está echo un grande lago.\\
El Lusitano.\\
Todo lo anda el buen Rey,\\
Dando muertes mui gallardo,\\
La espada tinta de sangre,\\
Lança rota, sin cavallo.\\
El Lusitano.\\
\\
Que el suyo passado el pecho\\
Ya no puede dar un passo,\\
A George Dalbiquerque pide\\
Le de su rucio rodado.\\
El Lusitano.\\
Daselo de buena gana,\\
Y el Rey cavalga de un salto,\\
Mirale el Rey como jaze,\\
De espaldas casi espirando.\\
El Lusitano.\\
\\
Mas le dize que se salve,\\
Pues todo es roto en pedaços,\\
Y el Rey se vá a los moros,\\
A los moros Sebastiano\\
El Lusitano.\\
Busca la muerte en dar muertes,\\
Sebastiano el Lusitano,\\
Diziendo aora es la hora,\\
Que un bel morir, tuta la vita honora.\\
\\
htmlnota: \\
Este romance referente à batalha de Alcácer-Quibir foi muito \\
divulgado e popular nos anos que se seguiram à batalha.\\
\\
A partitura a 3 vozes pode ser encontrada (por exemplo) no "Cancioneiro\\
de música popular portuguesa" de Giacometti.\\
\section{ Caixinhas}
Caixinhas,  sobre a colina,\\
Caixinhas de tic-tac\\
São caixinhas pequeninas,\\
Caixinhas todas iguais\\
Umas brancas, outras verdes,\\
Outras de cor azul,\\
Todas são de tic-tac\\
Iguaizinhas todas são\\
\\
A gente que mora nelas\\
Foi toda para a universidade\\
Onde os meteram em caixas\\
Ficaram todos iguais.\\
Engenheiros, advogados,\\
Médicos e administradores\\
Todos são de tic-tac\\
Iguaizinhos como são.\\
\\
Aos domingos vão ao futebol\\
E bebem Martini dry,\\
Têm filhos que são machos\\
E também hão-de estudar.\\
Ao findarem o liceu\\
Irão para a universidade,\\
Onde os vão meter em caixas\\
Vão ficar todos iguais.\\
\\
Os meninos vão para a vida\\
Casam-se e têm bébés\\
Em caixas, pequenas caixas,\\
Caixinhas todas iguais.\\
Umas brancas, outras verdes,\\
Outras de cor azul,\\
Todas são de tic-tac,\\
Iguaizinhas como são...\\
\section{ Canção para desfazer equívocos}
Irmão, irmão doutra cor,\\
Não estranhes que te chame irmão.\\
Tu lutas pela liberdade\\
Que os brancos não te querem dar.\\
E estranhas, que sendo um deles,\\
Eu venha chamar-te irmão.\\
\\
Irmão, irmão doutra cor,\\
Ouve o que quero dizer.\\
A nossa luta é a mesma,\\
Contra os homens que exploram homens\\
E não nos deixam ser homens,\\
Sejam eles brancos ou não.\\
\\
Irmão, irmão doutra cor,\\
Já vês porque te chamo irmão ?\\
Vamos dar as nossas mãos\\
E um dia que já não vem longe,\\
Os homens já saberão\\
Que somos irmãos.\\
\section{ Cavalo de várias cores}
Quero um cavalo de várias cores,\\
Quero-o depressa que vou partir.\\
Esperam-me prados com tantas flores,\\
Que só cavalos de várias cores\\
Podem servir.\\
\\
Quero uma sela feita de restos\\
Dalguma nuvem que ande no céu.\\
Quero-a evasiva - nimbos e cerros -\\
Sobre os valados, sobre os aterros,\\
Que o mundo é meu.\\
\\
Quero que as rédeas façam prodígios:\\
Voa, cavalo, galopa mais,\\
Trepa às camadas do céu sem fundo,\\
Rumo àquele ponto, exterior ao mundo,\\
Para onde tendem as catedrais.\\
\\
Deixem que eu parta, agora, já,\\
Antes que murchem todas as flores.\\
Tenho a loucura, sei o caminho,\\
Mas como posso partir sózinho\\
Sem um cavalo de várias cores ?\\
\section{ O menino negro não entrou na roda}
O menino negro não entrou na roda \\
das crianças brancas - as crianças brancas \\
que brincavam todas numa roda viva \\
de canções festivas, gargalhadas francas...\\
\\
O menino negro não entrou na roda.\\
\\
E chegou o vento junto das crianças\\
- e bailou com elas e cantou com elas \\
as canções e danças das suaves brisas, \\
as canções e danças das brutais procelas.\\
\\
O menino negro não entrou na roda.\\
\\
Pássaros, em bando, voaram chilreando \\
sobre as cabecinhas lindas dos meninos \\
e pousaram todos em redor. Por fim, \\
bailaram seus vôos, cantando seus hinos ...\\
\\
O menino negro não entrou na roda.\\
\\
"Venha cá, pretinho, venha cá brincar"\\
- disse um dos meninos com seu ar feliz.\\
A mamã, zelosa, logo fez reparo;\\
o menino branco já não quis, não quis ...\\
\\
o menino negro não entrou na roda.\\
\\
O menino negro não entrou na roda\\
das crianças brancas. Desolado, absorto, \\
ficou só, parado com olhar cego, \\
ficou só, calado com voz de morto.\\
\\
Nota:\\
Gravações: Luís Cília - 196? (??)\\
	   António Pedro Braga - 1970 (Movieplay SON 100.008)\\
\section{ Roda da senhora rainha}
Senhora rainha,\\
Deste seu reinado,\\
Escolha a linda prenda\\
Para o seu afilhado.\\
\\
Dos livros que tem,\\
Se lhe tem amor,\\
Dê-lhe todos eles\\
Que o faça doutor. \\
\\
Das terras que tem,\\
Desta sesmaria,\\
Dê-lhe todas elas\\
Vossa senhoria.\\
\\
Do ouro que tem,\\
Das jóias reais,\\
Dê-lhe todas elas\\
Não serão demais.\\
\\
Das armas que tem,\\
Vosso regimento,\\
Dê-lhe todas elas\\
Será o sargento.\\
\\
Do pão guardais\\
No real celeiro\\
Dê-lhe todo ele\\
Que é o primeiro.\\
\\
A coroa que tendes,\\
Do vosso mandar,\\
Ao vosso afilhado\\
A ide entregar.\\
\section{ Soldadim Catrapim}
Que linda menina, que vem de avental,\\
Criadinha linda tem o general.\\
Que lindo mancebo, de bota engraxada,\\
Que olhos que deita à linda criada.\\
\\
Ai soldado, soldadim, catrapim !\\
Minha mão não te dou não, catrapão !\\
Não te dou deste meu peito\\
A dor do meu coração.\\
\\
Que lindo navio, no cais a aprontar,\\
Adeus soldadinho, que não vais voltar.\\
Que linda menina, no cais a chorar,\\
Adeus soldadinho, que não vais voltar.\\
\\
Ai soldado, soldadim, catrapim !\\
Minha mão não te dou não, catrapão !\\
Não te dou deste meu peito\\
A dor do meu coração.\\
title : Aquarela do Brasil \\
singer: Gal Costa\\
author: Ary Barroso\\
from: Carlos Coutinho; José Roberto Molina\\
comm: canção escrita em 1939 \\
\\
Brasil, meu Brasil brasileiro \\
Meu mulato inzoneiro \\
Vou cantar-te nos meus versos \\
O Brasil, samba que dá \\
Bamboleio que faz gingar \\
O Brasil do meu amor \\
Terra de Nosso Senhor \\
Brasil, prá mim, prá mim, prá mim \\
\\
Abre a cortina do passado \\
Tira a mãe preta do serrado \\
Bota o Rei-Congo no congado \\
Brasil, prá mim \\
Deixa cantar de novo o trovador \\
A merencória luz da lua \\
Toda a canção do meu amor \\
Quero ver essa dona caminhando \\
Pelos salões arrastando \\
O seu vestido rendado \\
Brasil, prá mim, prá mim, prá mim \\
\\
Brasil, terra boa e gostosa \\
Da morena sestrosa \\
De olhar indiscreto \\
O Brasil, samba que dá \\
Bamboleio que faz gingar \\
O brasil do meu amor \\
Terra de Nosso Senhor \\
Brasil, prá mim, prá mim, prá mim \\
\\
Esse coqueiro que dá côco \\
Onde amarro a minha rede \\
Nas noites claras de luar \\
Brasil, prá mim \\
Ah ouve essas fontes murmurantes \\
Aonde eu mato a minha sede \\
E onde a lua vem brincar \\
Ah esse Brasil lindo e trigueiro \\
É o meu Brasil brasileiro \\
Terra de samba e pandeiro \\
Brasil, prá mim, prá mim, Brasil \\
Brasil, prá mim, prá mim, Brasil \\
Brasil, prá mim, prá mim quarela do Brasil \\
\\
htmlnota: \\
Em Novembro de 1997 esta canção foi votada como a Melhor Canção Brasileira \\
do Século por um jurado de treze peritos feito pela Academia Brasileira de \\
Letras. \\
\\
Esta canção tornou-se popular sob o nome "Brasil", tendo sido gravada inúmeras\\
vezes por vários artistas. \\
Dos seus mais conhecidos intérpretes destaca-se João Gilberto (com uma versão\\
ligeiramente alterada), Cármen Miranda, Gal Costa, etc. \\
\section{ Desfolhada}
Corpo de linho \\
lábios de mosto \\
meu corpo lindo \\
meu fogo posto. \\
Eira de milho \\
luar de Agosto \\
quem faz um filho \\
fá-lo por gosto. \\
É milho-rei \\
milho vermelho \\
cravo de carne \\
bago de amor \\
filho de um rei \\
que sendo velho \\
volta a nascer \\
quando há calor. \\
\\
Minha palavra dita à luz do sol nascente \\
meu madrigal de madrugada \\
amor amor amor amor amor presente \\
em cada espiga desfolhada. \\
\\
Minha raiz de pinho verde \\
meu céu azul tocando a serra \\
oh minha água e minha sede \\
oh mar ao sul da minha terra. \\
\\
É trigo loiro \\
é além tejo \\
o meu país \\
neste momento \\
o sol o queima \\
o vento o beija \\
seara louca em movimento. \\
\\
Minha palavra dita à luz do sol nascente \\
meu madrigal de madrugada \\
amor amor amor amor amor presente \\
em cada espiga desfolhada. \\
\\
Olhos de amêndoa \\
cisterna escura \\
onde se alpendra \\
a desventura. \\
Moira escondida \\
moira encantada \\
lenda perdida \\
lenda encontrada. \\
Oh minha terra \\
minha aventura \\
casca de noz \\
desamparada. \\
Oh minha terra \\
minha lonjura \\
por mim perdida \\
por mim achada. \\
\\
htmlnota:\\
Música de Nuno Nazareth Fernandes. Escrita em 1968. Foi\\
inicialmente patenteada com o título <i>Desfolhada Portuguesa</i>,\\
modificado pelo autor em 1969 para <i>Desfolhada</i>.\\
Interpretada por Simone de Oliveira, concorreu ao Festival da RTP\\
em 1969, obtendo o 1º lugar. Interpretada por Simone de Oliveira\\
no disco Valentim de Carvalho PEP 1276. </p>\\
\section{ Meu amor, meu amor}
Meu amor meu amor\\
meu corpo em movimento\\
minha voz à procura\\
do seu próprio lamento.\\
\\
Meu limão de amargura meu punhal a escrever\\
nós parámos o tempo não sabemos morrer\\
e nascemos nascemos\\
do nosso entristecer.\\
\\
Meu amor meu amor\\
meu nó e sofrimento\\
minha mó de ternura\\
minha nau de tormento\\
\\
este mar não tem cura este céu não tem ar\\
nós parámos o vento não sabemos nadar\\
e morremos morremos\\
devagar devagar.\\
\section{ Quando um Homem Quiser}
Tu que dormes a noite na calçada de relento\\
Numa cama de chuva com lençóis feitos de vento\\
Tu que tens o Natal da solidão, do sofrimento\\
És meu irmão amigo\\
És meu irmão\\
 \\
E tu que dormes só no pesadelo do ciúme\\
Numa cama de raiva com lençóis feitros de lume\\
E sofres o Natal da solidão sem um queixume\\
És meu irmão amigo\\
És meu irmão\\
 \\
Natal é em Dezembro\\
Mas em Maio pode ser\\
Natal é em Setembro\\
É quando um homem quiser\\
Natal é quando nasce uma vida a amanhecer\\
Natal é sempre o fruto que há no ventre da Mulher\\
 \\
Tu que inventas ternura e brinquedos para dar\\
Tu que inventas bonecas e combóios de luar\\
E mentes ao teu filho por não os poderes comprar\\
És meu irmão amigo\\
És meu irmão\\
 \\
E tu que vês na montra a tua fome que eu não sei\\
Fatias de tristeza em cada alegre bolo-rei\\
Pões um sabor amargo em cada doce que eu comprei\\
És meu irmão amigo\\
És meu irmão\\
 \\
Natal é em Dezembro\\
Mas em Maio pode ser\\
Natal é em Setembro\\
É quando um homem quiser\\
Natal é quando nasce uma vida a amanhecer\\
Natal é sempre o fruto que há no ventre da Mulher\\
\section{ Tourada}
Não importa sol ou sombra \\
camarotes ou barreiras \\
toureamos ombro a ombro \\
as feras. \\
Ninguém nos leva ao engano \\
toureamos mano a mano \\
só nos podem causar dano \\
espera. \\
\\
Entram guizos chocas e capotes \\
e mantilhas pretas \\
entram espadas chifres e derrotes \\
e alguns poetas \\
entram bravos cravos e dichotes \\
porque tudo o mais \\
são tretas. \\
\\
Entram vacas depois dos forcados \\
que não pegam nada. \\
Soam brados e olés dos nabos \\
que não pagam nada \\
e só ficam os peões de brega \\
cuja profissão \\
não pega. \\
\\
Com bandarilhas de esperança \\
afugentamos a fera \\
estamos na praça \\
da Primavera. \\
\\
Nós vamos pegar o mundo \\
pelos cornos da desgraça \\
e fazermos da tristeza \\
graça. \\
\\
Entram velhas doidas e turistas \\
entram excursões \\
entram benefícios e cronistas \\
entram aldrabões \\
entram marialvas e coristas \\
entram galifões \\
de crista. \\
\\
Entram cavaleiros à garupa \\
do seu heroísmo \\
entra aquela música maluca \\
do passodoblismo \\
entra a aficionada e a caduca \\
mais o snobismo \\
e cismo... \\
\\
Entram empresários moralistas \\
entram frustrações \\
entram antiquários e fadistas \\
e contradições \\
e entra muito dólar muita gente \\
que dá lucro as milhões. \\
\\
E diz o inteligente \\
que acabaram asa canções. \\
\\
htmlnote:\\
Musica de Fernando Tordo.  Escrito no final de 1972.\\
Interpretada por Fernando Tordo, concorreu ao Festival da RTP de\\
1973 onde obteve o 1º lugar. Interpretado por Fernando Tordo no\\
disco TECLA TE 20060.\\
\section{ Ai se a Luzia}
Ai se a Luzia (x4)\\
\\
Ai se a Luzia soubesse que vinhas\\
Ai como depois tu vieste\\
Ai mandaria soprar pela rua\\
Ai três vendavais lá do Nordeste\\
\\
Ai se a Luzia um dia soubesse\\
Ai se a Luzia um dia sonhasse\\
Ai se a Luzia um dia a saber viesse\\
Melhor seria que um mau caruncho em mim entrasse\\
\\
Ai se a Luzia sonhasse a metade\\
Ai das coisas que nós fizemos\\
Ai rogaria um livro de pragas\\
Ai uma antologia dos demos\\
\\
Ai se a Luzia um dia soubesse...\\
\\
Ai se a Luzia contasse os cabelos \\
Ai que de ti tenho guardados\\
Ai não chegaria nem a vida inteira\\
Ai para tê-los todos contados\\
\\
Ai se a Luzia um dia soubesse...\\
\section{ * Canto de amor e trabalho}
[arre burra]\\
_Já lá vai o sol                   E\\
Já lá vai o dia\\
[anda bonita, anda burra]\\
Já me chega a noite\\
Já se vê a aldeia\\
[é bonita toma lá mais rédea]\\
[ah que nos doi o corpo]\\
A meranda é pouca\\
E o trabalho e o trabalho é duro\\
A mulher á espera\\
Já se fez noitinha\\
E a menina é já dormindo\\
E a menina está dormindo\\
\\
O teu pai vem do trabalho\\
Meu amor vem da campina\\
Ao chegar o caldinho ao burralho\\
Para não acordar a menina\\
\\
[arre burra é bonita vamos embora]\\
Ai o frio já aperta\\
Vai-se o verão vem o inverno\\
[arreda que vai doido brrr]\\
Terra brava terra farta\\
Terra que te quero bem\\
Terra que te quero bem\\
[eh bonita dá mais rédea, ah já se vê a casa]\\
Eh a ceia no braseiro\\
Já lhe sinto o gosto\\
Já lhe sinto o cheiro\\
E a mulher à espera\\
E a menina está dormindo\\
[oh bonita]\\
Ai a menina está dormindo\\
\\
O teu pai vem do trabalho\\
Meu amor vem da campina\\
Ao chegar o caldinho ao burralho\\
Para não acordar a menina\\
\\
\section{ É ouvi-los}
É ouvi-los É ouvi-los É ouvi-los\\
É ouvi-los É ouvi-los É ouvi-los\\
É ouvi-los numa faina\\
de enganar o pé\\
ai numa faina\\
de enganar o pão\\
\\
É ouvi-los É ouvi-los É ouvi-los\\
É ouvi-los É ouvi-los É ouvi-los\\
É ouvi-los numa teima\\
de enganar o Zé\\
ai numa teima\\
de enganar o Zé até mais não\\
\\
É ouvi-los É ouvi-los É ouvi-los\\
É ouvi-los É ouvi-los É ouvi-los\\
É ouvi-los numa farda\\
com espingarda ai é\\
ai numa farsa\\
com espingarda na mão\\
\\
É ouvi-los É ouvi-los É ouvi-los\\
É ouvi-los É ouvi-los É ouvi-los\\
É ouvi-los numa treta\\
de pasmar até\\
ai numa treta\\
de pasmar até um cão\\
\section{ Horas de ponta e mola}
É um sujeito é um escritório\\
uma gravata, um suspensório\\
uma conversa de latrina\\
é um verbete, uma aldrabice\\
é um trabalho , uma chatice\\
entre fumo e aspirina\\
\\
É numa rua o pôr da sola\\
calçada nas horas de ponta e mola\\
são conversas de cotovelo\\
é um eléctrico um pendura\\
um regresso e uma tontura\\
é um sorrir muito amarelo\\
\\
É uma casa uma família\\
uma torrada um chá de tília\\
uma conversa de fastídio\\
é um chinelo e um menino\\
televisão com o Vitorino\\
a lentidão de um suicídio\\
\\
É numa rua o pôr da sola\\
calçada nas horas de ponta e mola\\
é um silencio e um ritual\\
são os lacaios do comendador\\
são as gravatas sem côr\\
na procissão dum funeral\\
\\
\section{ Natação obrigatória}
Viemos do fundapique\\
passámos no tudasaque\\
não há mal que mal nos fique\\
nem há cu que não dê traque\\
mal a gente vem ao mundo\\
logo a gente vai ao fundo\\
\\
Andámos no malsalgado\\
brigámos no daceleste\\
e o escorbuto mal curado\\
com tratamento indigesto\\
mal a gente vem ao mundo\\
logo a gente vai ao fundo\\
\\
{soc}\\
Natação obrigatória\\
na introdução à instrução primária\\
natação obrigatória\\
para a salvação é condição necessária\\
não há cu que não dê traque\\
não há cu que não dê traque\\
mal a gente vem ao mundo\\
logo a gente vai ao fundo\\
{eoc}\\
\\
Pusemos a cachimónia\\
em papas de sarrabulho\\
e quando as noites são de insónia\\
damos voltas ao entulho\\
mal a gente vem ao mundo\\
logo a gente vai ao fundo\\
\\
Aprendizes da política\\
só na tática do "empocha"\\
vem a tempestade mítica\\
e s cabeça dá na rocha\\
mal a gente vem ao mundo\\
logo a gente vai ao fundo\\
\\
[refrão]\\
